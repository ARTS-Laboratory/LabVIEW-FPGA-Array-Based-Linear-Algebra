\documentclass[11pt,letterpaper,twocolumn]{article}

\usepackage{mathptmx} % added for time new roman font
\usepackage[left=1in,right=1in,top=1in,bottom=1in]{geometry}

\usepackage[latin1]{inputenc}
\usepackage{amsmath}
\usepackage{amsfonts}
\usepackage{amssymb}
\usepackage{graphicx}
\usepackage{booktabs}        % ADDED FOR TABLES    
\usepackage{placeins}        % Added for \FloatBarrier
\usepackage{enumitem}        % Added for changes in \item
\usepackage{commath}         % added for \norm 
\usepackage{wrapfig}
\usepackage{tabu}
%set page header and footer

\usepackage{titlesec}
\titlespacing*{\section}
{0pt}{1.5ex}{1ex}
\titlespacing*{\subsection}
{0pt}{1ex}{0.8ex}
\titlespacing*{\subsubsection}
{0pt}{0.8ex}{0.4ex}

\usepackage{color} % color added for editing
\newcommand{\bl}[1]{\textcolor[rgb]{0.00,0.00,1.00}{#1}}
\newcommand{\gr}[1]{\textcolor[rgb]{0.00,0.50,0.00}{#1}}
\newcommand{\rd}[1]{\textcolor[rgb]{0.75,0.00,0.00}{#1}}

\title{Generalized Eigenvalue Problem}
\date{}


\begin{document}

\maketitle

\noindent The generalized eigenvalue problem of two symmetric matrices  ${\bf K}={\bf K}^T$ and ${\bf M}={\bf M}^T$:

\begin{equation}
{\bf K}\Phi=\lambda{\bf M}\Phi
\end{equation}

\noindent Cholesky factorization leads to: 
\begin{equation}
{\bf M} = {\bf L}{\bf L}^T
\end{equation}
The  generalized eigenvalue problem can be reduced to:

\begin{equation}
{\bf C}y = \lambda y
\end{equation}
\noindent where: 
\begin{equation}
{\bf C} ={\bf L}^{-1} {\bf K} {\bf L}^{-T}
\end{equation}

Given that ${\bf C}$ is a symmetric matrix, its eigenvalues can be solved for using the Exact Jacobi Method. This is done by reducing the norm of the off diagonal elements:
\begin{equation}
F({\bf C}) = \sqrt{\sum_{i=1}^{n}\sum_{j=1,j \ne i}^{n}c_{ij}^2 }
\end{equation}
\noindent This is achomplished by a sequence of orhogonal similarity transformations:
\begin{equation}
{\bf C}^{(K+1)} = {\bf J}^T_{pq}{\bf C}^K{\bf J}_{pq}, \hspace{1cm} k=0,1,2,...
\end{equation}
\noindent where:
\begin{equation}
{\bf C}^{0} = {\bf C}
\end{equation}



%
%\pagebreak
%The generalized eigenvalue problem of two symmetric matrices  ${\bf A}={\bf A}^T$ and ${\bf B}={\bf B}^T$ is to find a scalar $\lambda$ and the corresponding vector $\Phi$ for the following equation to hold: 
%
%\begin{equation}
%{\bf A}\Phi_i=\lambda_i{\bf B}\Phi_i, \;\;\;\;\;\;\;(i=1,\cdots,n)
%\end{equation}
%
%\noindent or in matrix form 
%\begin{equation}
%{\bf A}\Phi={\bf B}\Phi\Lambda
%\end{equation}
%
%\noindent where the matrixies represent
%
%\begin{itemize}
%\item $\Lambda$ = eigenvalue
%\item $\Phi$ = eigenvector
%\item \textbf{A} = \textbf{K} matirx in the final problem
%\item \textbf{B} = \textbf{M} matirx in the final problem
%\end{itemize}
%
% The eigenvalue and eigenvector matrices $\Lambda$ and $\Phi$ can be found in the following steps:
%
%\section{Solve the eigenvalue problem of ${\bf B}$}
%Solve the eigenvalue problem of ${\bf B}$ to find its diagonal eigenvalue matrix $\Lambda_B$ and orthogonal eigenvector matrix  
%\begin{equation}
%\Phi_B=(\Phi_B^T)^{-1}
%\end{equation}
%\noindent so that 
%\begin{equation}
%{\bf B}\Phi_B=\Phi_B\Lambda_B
%\end{equation}
%\noindent or 
%\begin{equation}
%\Phi_B^{-1}{\bf B}\Phi_B = \Phi_B^{T}{\bf B}\Phi_B = \Lambda_B
%\end{equation}
%
%
%\section{Multiplying both sides of the second equation above by  $\Lambda^{-1/2}$}
%
%Left and right multiplying both sides of the second equation above by  $\Lambda^{-1/2}$ (whitening) we get 
%\begin{equation}
%\Lambda^{-1/2}_B (\Phi_B^T{\bf B}\Phi_B)\Lambda_B^{-1/2}      =    \Lambda^{-1/2}_B    \Lambda_B\Lambda^{-1/2}_B={\bf I}
%\end{equation}
%
%\noindent We define 
%\begin{equation}
%\Phi'_B=\Phi_B\Lambda^{-1/2}_B
%\end{equation}
%
%\noindent and get 
%\begin{equation}
%(\Phi'_B)^T{\bf B}\Phi'_B={\bf I}
%\end{equation}
%
%\noindent Note that $\Phi_B'$ is not orthogonal 
%\begin{equation}
%(\Phi')^{-1}_B=(\Phi_B\Lambda^{-1/2}_B)^{-1} = \Lambda_B^{1/2}\Phi_B^{-1}  = \Lambda_B^{1/2}\Phi_B^{T} \ne \Lambda^{-1/2}_B\Phi^T_B=\Phi^T_B
%\end{equation}
%
%\section{Apply the same transform to ${\bf A}$}
%Apply the same transform to ${\bf A}$: 
%\begin{equation}
%(\Phi'_B)^T{\bf A}\Phi'_B =(\Lambda^{-1/2}_B \Phi_B^T){\bf A}(\Phi_B\Lambda^{-1/2}_B) ={\bf A}'
%\end{equation}
%
%\noindent Note that ${\bf A}'$ is symmetric as well as ${\bf A}$: 
%\begin{equation}
%{\bf A}'^T = (\Phi_B'^T {\bf A} \Phi_B')^T =\Phi_B'^T{\bf A}\Phi'_B={\bf A}' 
%\end{equation}
%
%\section{Diagonalize ${\bf A}'$}
%As ${\bf A}'$ is symmetric,, it can be diagonalized by its orthogonal eigenvector matrix $\Phi_A$: 
%
%\begin{equation}
%\Phi_A^T{\bf A}'\Phi_A=\Lambda
%\end{equation}
%
%\noindent i.e., 
%
%\begin{equation}
%\Phi_A^T (\Lambda^{-1/2}_B\Phi_B^T{\bf A}\Phi_B\Lambda^{-1/2}_B)\Phi_A	 = (\Phi_A^T \Lambda^{-1/2}_B\Phi_B^T){\bf A}( \Phi_B\Lambda^{-1/2}_B \Phi_A) = \Phi^T {\bf A}\Phi =\Lambda
%\end{equation}
%
%\noindent where we have defined 
%\begin{equation}
%\Phi=\Phi_B\Lambda^{-1/2}_B\Phi_A
%\end{equation}
%
%\noindent which is not orthogonal: 
%\begin{equation}
%\Phi^{-1}=(\Phi_B\Lambda^{-1/2}_B\Phi_A)^{-1} = \Phi_A^T\Lambda^{-1/2}_B\Phi_B^{-1}=\Phi_A^T\Lambda^{1/2}_B\Phi_B^T \ne \Phi_A^T\Lambda^{-1/2}_B\Phi_B^T = \Phi^T
%\end{equation}
%
%\section{This $\Phi$ also diagonalizes ${\bf B}$: }
%This $\Phi$ also diagonalizes ${\bf B}$: 
%
%\begin{equation}
%\Phi^T{\bf B}\Phi =  (\Phi_B\Lambda^{-1/2}_B\Phi_A)^T {\bf B}(\Phi_B\Lambda_B^{-1/2}\Phi_A) = \Phi^T_A (\Phi_B^T\Lambda^{-1/2}_B{\bf B}\Phi_B)\Lambda^{-1/2}_B\Phi_A = \Phi_A^T\Lambda^{-1/2}_B\Lambda_B\Lambda^{-1/2}_B\Phi_A =\Phi_A^T\Phi_A={\bf I}
%\end{equation}
%
%\section{Now we have}
%Now we have 
%\begin{equation}
%\left\{ \begin{array}{l} \Phi^T{\bf A}\Phi=\Lambda\\
%\Phi^T{\bf B}\Phi={\bf I}\end{array}\right.
%\end{equation}
%
%\noindent Right multiplying both sides of the second equation by $\Lambda$ and equating the left-hand side to that of the first equation, we get 
%\begin{equation}
%{\bf A}\Phi={\bf B}\Phi\Lambda
%\end{equation}
%
%\noindent i.e., $\Lambda$ and $\Phi$ are the eigenvalue and eigenvector matrices of the generalized eigenvalue problem. Note, however, as shown above, $\Phi$ is not orthogonal.
%The Rayleigh quotient of two symmetric matrices ${\bf A}$ and ${\bf B}$ is a function of a vector ${\bf w}$ defined as: 
%
%\begin{equation}
%R({\bf w})=\frac{{\bf w}^T{\bf A}{\bf w}}{{\bf w}^T{\bf B}{\bf w}}
%\end{equation}
%
%To find the optimal ${\bf w}$ corresponding to the extremum (maximum or minimum) of $R({\bf w})$, we find its derivative with respect to ${\bf w}$: 
%\begin{equation}
%\frac{d}{d{\bf w}}R({\bf w}) =\frac{2{\bf A}{\bf w}({\bf w}^T{\bf B}{\bf w}) -2 {\bf B}{\bf w} ({\bf w}^T{\bf A}{\bf w})}{({\bf w}^T{\bf B}{\bf w})^2}
%\end{equation}
%
%\noindent Setting it to zero we get 
%\begin{equation}
%{\bf A}{\bf w}({\bf w}^T{\bf B}{\bf w})={\bf B}{\bf w}({\bf w}^T{\bf A}{\bf w}) 
%\end{equation}
%
%\noindent i.e.,
%
%\begin{equation}
% {\bf A}{\bf w} \frac{{\bf w}^T{\bf A}{\bf w}}{{\bf w}^T{\bf B}{\bf w}}   {\bf B}{\bf w} = R({\bf w}){\bf B}{\bf w}=\lambda {\bf B}{\bf w}
%\end{equation}
%
%
%\noindent The second equation can be recognized as a generalized eigenvalue problem with  $\lambda=R({\bf w})$ being the eigenvalue and and ${\bf w}$ the corresponding eigenvector. Solving this we get the vector ${\bf w}=\Phi$ corresponding to the maximum/minimum eigenvalue $\lambda=R({\bf w})$, which maximizes/minimizes the Rayleigh quotient.








\end{document}
